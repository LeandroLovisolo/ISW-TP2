\documentclass[a4paper, 10pt, twoside]{article}

\usepackage[top=1in, bottom=1in, left=1in, right=1in]{geometry}
\usepackage[utf8]{inputenc}
\usepackage[spanish, es-ucroman, es-noquoting]{babel}
\usepackage{setspace}
\usepackage{fancyhdr}
\usepackage{lastpage}
\usepackage{amsmath}
\usepackage{amsfonts}
\usepackage{amsthm}
\usepackage{verbatim}
\usepackage{graphicx}
\usepackage{float}
\usepackage{enumitem} % Provee macro \setlist
\usepackage{tabularx}
\usepackage{multirow}
\usepackage{hyperref}
\usepackage{multicol}
\usepackage{pbox}
\usepackage[toc, page]{appendix}


%%%%%%%%%% Configuración de Fancyhdr - Inicio %%%%%%%%%%
\pagestyle{fancy}
\thispagestyle{fancy}
\lhead{Trabajo Práctico 2 · Ingeniería de Software I}
\rhead{Delgado · Lovisolo · Petaccio · Requeni · Vita}
\renewcommand{\footrulewidth}{0.4pt}
\cfoot{\thepage /\pageref{LastPage}}

\fancypagestyle{caratula} {
   \fancyhf{}
   \cfoot{\thepage /\pageref{LastPage}}
   \renewcommand{\headrulewidth}{0pt}
   \renewcommand{\footrulewidth}{0pt}
}
%%%%%%%%%% Configuración de Fancyhdr - Fin %%%%%%%%%%


%%%%%%%%%% Miscelánea - Inicio %%%%%%%%%%
% Evita que el documento se estire verticalmente para ocupar el espacio vacío
% en cada página.
\raggedbottom

% Deshabilita sangría en la primer línea de un párrafo.
\setlength{\parindent}{0em}

% Separación entre párrafos.
\setlength{\parskip}{0.5em}

% Separación entre elementos de listas.
\setlist{itemsep=0.5em}

% Asigna la traducción de la palabra 'Appendices'.
\renewcommand{\appendixtocname}{Apéndices}
\renewcommand{\appendixpagename}{Apéndices}
%%%%%%%%%% Miscelánea - Fin %%%%%%%%%%


%%%%%%%%%% Insertar diagrama - Inicio %%%%%%%%%%
\newcommand{\diagramav}[1]{
  \includegraphics[type=png,ext=.png,read=.png,width=16cm]{diagramas/#1}
}

\newcommand{\diagramah}[1]{
  \includegraphics[type=png,ext=.png,read=.png,height=16cm,angle=90]{diagramas/#1}
}
%%%%%%%%%% Insertar diagrama - Fin %%%%%%%%%%

\begin{document}


%%%%%%%%%%%%%%%%%%%%%%%%%%%%%%%%%%%%%%%%%%%%%%%%%%%%%%%%%%%%%%%%%%%%%%%%%%%%%%%
%% Carátula                                                                  %%
%%%%%%%%%%%%%%%%%%%%%%%%%%%%%%%%%%%%%%%%%%%%%%%%%%%%%%%%%%%%%%%%%%%%%%%%%%%%%%%


\thispagestyle{caratula}

\begin{center}

\includegraphics[height=2cm]{DC.png} 
\hfill
\includegraphics[height=2cm]{UBA.jpg} 

\vspace{2cm}

Departamento de Computación,\\
Facultad de Ciencias Exactas y Naturales,\\
Universidad de Buenos Aires

\vspace{4cm}

\begin{Huge}
Trabajo Práctico 2
\end{Huge}

\vspace{0.5cm}

\begin{Large}
Ingeniería de Software I
\end{Large}

\vspace{1cm}

Primer Cuatrimestre de 2014

\vspace{4cm}

\begin{tabular}{|c|c|c|}
\hline
Apellido y Nombre & LU & E-mail\\
\hline
Delgado, Alejandro N.  & 601/11 & nahueldelgado@gmail.com\\
Lovisolo, Leandro      & 645/11 & leandro@leandro.me\\
Petaccio, Lautaro José & 443/11 & lausuper@gmail.com\\
Requeni, Gastón        & 400/11 & grequeni@hotmail.com\\
Vita, Sebastián        & 149/11 & sebastian\_vita@yahoo.com.ar\\
\hline
\end{tabular}

\end{center}

\newpage


%%%%%%%%%%%%%%%%%%%%%%%%%%%%%%%%%%%%%%%%%%%%%%%%%%%%%%%%%%%%%%%%%%%%%%%%%%%%%%%
%% Índice                                                                    %%
%%%%%%%%%%%%%%%%%%%%%%%%%%%%%%%%%%%%%%%%%%%%%%%%%%%%%%%%%%%%%%%%%%%%%%%%%%%%%%%


\tableofcontents

\newpage


%%%%%%%%%%%%%%%%%%%%%%%%%%%%%%%%%%%%%%%%%%%%%%%%%%%%%%%%%%%%%%%%%%%%%%%%%%%%%%%
%% Introducción                                                              %%
%%%%%%%%%%%%%%%%%%%%%%%%%%%%%%%%%%%%%%%%%%%%%%%%%%%%%%%%%%%%%%%%%%%%%%%%%%%%%%%


\section{Introducción}


\newcounter{usecasecounter}
\newcounter{usecasealtcounter}

\newcommand{\ucname}[1]{\renewcommand{\givenucname}{#1}}
\newcommand{\ucpre}[1]{\renewcommand{\givenucpre}{#1}}
\newcommand{\ucpost}[1]{\renewcommand{\givenucpost}{#1}}
\newcommand{\ucactor}[1]{\renewcommand{\givenucactor}{#1}}
\newcommand{\givenucname}{REQUIRED!}
\newcommand{\givenucpre}{REQUIRED!}
\newcommand{\givenucpost}{REQUIRED!}
\newcommand{\givenucactor}{REQUIRED!}

\newenvironment{usecase}
  {}{
    \textbf{Caso de uso: }\givenucname \\
    \textbf{Pre: }\givenucpre \\
    \textbf{Post: }\givenucpost \\
    \textbf{Actores: }\givenucactor
  }

\newenvironment{usecasesteps}
  {
    \setcounter{usecasecounter}{0}
    \setcounter{usecasealtcounter}{0}

    \begin{tabular}{|l|l|}
    \hline
    Curso normal & Curso alternativo \\
    \hline
    \hline
  }{
    \end{tabular}
  }

\newcommand{\ucstep}[2]{
  \stepcounter{usecasecounter}
  \setcounter{usecasealtcounter}{0}
  \makebox[4ex][l]{\arabic{usecasecounter}.} #1 & \pbox[t]{5cm}{#2} \\
  \hline
}

\newcommand{\ucalt}[1]{
  \stepcounter{usecasealtcounter}
  \makebox[4ex][l]{\arabic{usecasecounter}.\arabic{usecasealtcounter}.}
  #1
}

\begin{usecase}
  \ucpre{Terminó la secundaria}
  \ucpost{Recibe premio Turing}
  \ucname{Ser un mostro}
  \ucactor{Leandro}
\end{usecase}
\begin{usecasesteps}
  \ucstep{Estudiar computación.}
         {\ucalt{Dedicarse a bellas artes.} \\
          \ucalt{Morirse de hambre.}}
  \ucstep{Resolver $P=NP$.}
         {\ucalt{Trabajar en otro problema.}}
  \ucstep{Recibir premio Turing.}{}
\end{usecasesteps}


\vspace{2cm}




%%%%%%%%%%%%%%%%%%%%%%%%%%%%%%%%%%%%%%%%%%%%%%%%%%%%%%%%%%%%%%%%%%%%%%%%%%%%%%%
%% Desarrollo                                                                %%
%%%%%%%%%%%%%%%%%%%%%%%%%%%%%%%%%%%%%%%%%%%%%%%%%%%%%%%%%%%%%%%%%%%%%%%%%%%%%%%


\section{Desarrollo}


%%%%%%%%%%%%%%%%%%%%%%%%%%%%%%%%%%%%%%%%%%%%%%%%%%%%%%%%%%%%%%%%%%%%%%%%%%%%%%%
%% Casos de uso                                                              %%
%%%%%%%%%%%%%%%%%%%%%%%%%%%%%%%%%%%%%%%%%%%%%%%%%%%%%%%%%%%%%%%%%%%%%%%%%%%%%%%


\section{Casos de uso}

\subsection{Registrando retiro de bicicleta}

\textbf{Caso de uso:} Registrando retiro de bicicleta

\textbf{Pre:} True

\textbf{Post:} Se registra el retiro de bicicleta

\textbf{Actores:} Personal de la estación
\\

\begin{tabular}{| p{7cm} | p{7cm} |}
	\hline
	Caso de uso normal & Caso de uso alternativo \\ \hline
	1) El sistema registra la petición de una bicicleta &  \\ \hline
	2) El sistema consulta el stock & \\ \hline
	3) El sistema reserva una bicicleta del stock hasta el fin del CU & 3.1) Si no hay stock, muestra que no hay stock. Fin CU \\ \hline
	4) El personal de la estación ingresa el DNI & \\ \hline
	5) El sistema verifica que el usuario esté registrado & \\ \hline
	6) El sistema verifica que el usuario no esté penalizado & 6.1) Si el usuario no está registrado, muestra que no existe en el sistema. Fin CU \\ \hline
	7) El personal de la estación ingresa el número de la bicicleta a asignar al usuario. & 7.1) Si el usuario está penalizado, se informa que lo está. Fin CU \\ \hline
	8) El sistema registra la entrega de la bicicleta guardando DNI, ID de bicicleta y fecha y hora actual. Fin CU & \\
	\hline
\end{tabular}

\end{document}