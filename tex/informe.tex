\documentclass[a4paper, 10pt, twoside]{article}

\usepackage[top=1in, bottom=1in, left=1in, right=1in]{geometry}
\usepackage[utf8]{inputenc}
\usepackage[spanish, es-ucroman, es-noquoting]{babel}
\usepackage{setspace}
\usepackage{fancyhdr}
\usepackage{lastpage}
\usepackage{amsmath}
\usepackage{amsfonts}
\usepackage{amsthm}
\usepackage{verbatim}
\usepackage{graphicx}
\usepackage{float}
\usepackage{enumitem} % Provee macro \setlist
\usepackage{tabularx}
\usepackage{multirow}
\usepackage{hyperref}
\usepackage{multicol}
\usepackage{pbox}
\usepackage[toc, page]{appendix}


%%%%%%%%%% Configuración de Fancyhdr - Inicio %%%%%%%%%%
\pagestyle{fancy}
\thispagestyle{fancy}
\lhead{Trabajo Práctico 2 · Ingeniería de Software I}
\rhead{Delgado · Lovisolo · Petaccio · Requeni · Vita}
\renewcommand{\footrulewidth}{0.4pt}
\cfoot{\thepage /\pageref{LastPage}}

\fancypagestyle{caratula} {
   \fancyhf{}
   \cfoot{\thepage /\pageref{LastPage}}
   \renewcommand{\headrulewidth}{0pt}
   \renewcommand{\footrulewidth}{0pt}
}
%%%%%%%%%% Configuración de Fancyhdr - Fin %%%%%%%%%%


%%%%%%%%%% Miscelánea - Inicio %%%%%%%%%%
% Evita que el documento se estire verticalmente para ocupar el espacio vacío
% en cada página.
\raggedbottom

% Deshabilita sangría en la primer línea de un párrafo.
\setlength{\parindent}{0em}

% Separación entre párrafos.
\setlength{\parskip}{0.5em}

% Separación entre elementos de listas.
\setlist{itemsep=0.5em}

% Asigna la traducción de la palabra 'Appendices'.
\renewcommand{\appendixtocname}{Apéndices}
\renewcommand{\appendixpagename}{Apéndices}
%%%%%%%%%% Miscelánea - Fin %%%%%%%%%%


%%%%%%%%%% Insertar diagrama - Inicio %%%%%%%%%%
\newcommand{\diagramav}[1]{
  \includegraphics[type=png,ext=.png,read=.png,width=16cm]{diagramas/#1}
}

\newcommand{\diagramah}[1]{
  \includegraphics[type=png,ext=.png,read=.png,height=16cm,angle=90]{diagramas/#1}
}
%%%%%%%%%% Insertar diagrama - Fin %%%%%%%%%%


%%%%%%%%%% Macros para Casos de Uso - Inicio %%%%%%%%%%
\newcounter{usecasecounter}
\newcounter{usecasealtcounter}

\newcommand{\ucname}[1]{\renewcommand{\givenucname}{#1}}
\newcommand{\ucpre}[1]{\renewcommand{\givenucpre}{#1}}
\newcommand{\ucpost}[1]{\renewcommand{\givenucpost}{#1}}
\newcommand{\ucactor}[1]{\renewcommand{\givenucactor}{#1}}
\newcommand{\givenucname}{REQUIRED!}
\newcommand{\givenucpre}{REQUIRED!}
\newcommand{\givenucpost}{REQUIRED!}
\newcommand{\givenucactor}{REQUIRED!}

\newenvironment{usecase}
  {}{
    \textbf{Caso de uso: }\givenucname \\
    \textbf{Pre: }\givenucpre \\
    \textbf{Post: }\givenucpost \\
    \textbf{Actores: }\givenucactor
  }

\newenvironment{usecasesteps}
  {
    \setcounter{usecasecounter}{0}
    \setcounter{usecasealtcounter}{0}

    \tabularx{\textwidth}{|l|X|}
    \hline
    Curso normal & Curso alternativo \\
    \hline
    \hline
  }{
    \endtabularx
    \vspace{\parskip}
  }

\newcommand{\ucstep}[2]{
  \stepcounter{usecasecounter}%
  \parbox[t]{7.5cm}{
    \makebox[2ex][l]{\arabic{usecasecounter}.}
    #1\phantom{p}
  }
  &
  \parbox[t]{7.5cm}{
    \setcounter{usecasealtcounter}{0}
    #2\phantom{p}
  } \\
  \hline
}

\newcommand{\ucalt}[1]{
  \stepcounter{usecasealtcounter}
  \makebox[4ex][l]{\arabic{usecasecounter}.\arabic{usecasealtcounter}.}
  #1
}
%%%%%%%%%% Macros para Casos de Uso - Fin %%%%%%%%%%


\begin{document}


%%%%%%%%%%%%%%%%%%%%%%%%%%%%%%%%%%%%%%%%%%%%%%%%%%%%%%%%%%%%%%%%%%%%%%%%%%%%%%%
%% Carátula                                                                  %%
%%%%%%%%%%%%%%%%%%%%%%%%%%%%%%%%%%%%%%%%%%%%%%%%%%%%%%%%%%%%%%%%%%%%%%%%%%%%%%%


\thispagestyle{caratula}

\begin{center}

\includegraphics[height=2cm]{DC.png} 
\hfill
\includegraphics[height=2cm]{UBA.jpg} 

\vspace{2cm}

Departamento de Computación,\\
Facultad de Ciencias Exactas y Naturales,\\
Universidad de Buenos Aires

\vspace{4cm}

\begin{Huge}
Trabajo Práctico 2
\end{Huge}

\vspace{0.5cm}

\begin{Large}
Ingeniería de Software I
\end{Large}

\vspace{1cm}

Primer Cuatrimestre de 2014

\vspace{4cm}

\begin{tabular}{|c|c|c|}
\hline
Apellido y Nombre & LU & E-mail\\
\hline
Delgado, Alejandro N.  & 601/11 & nahueldelgado@gmail.com\\
Lovisolo, Leandro      & 645/11 & leandro@leandro.me\\
Petaccio, Lautaro José & 443/11 & lausuper@gmail.com\\
Requeni, Gastón        & 400/11 & grequeni@hotmail.com\\
Vita, Sebastián        & 149/11 & sebastian\_vita@yahoo.com.ar\\
\hline
\end{tabular}

\end{center}

\newpage


%%%%%%%%%%%%%%%%%%%%%%%%%%%%%%%%%%%%%%%%%%%%%%%%%%%%%%%%%%%%%%%%%%%%%%%%%%%%%%%
%% Índice                                                                    %%
%%%%%%%%%%%%%%%%%%%%%%%%%%%%%%%%%%%%%%%%%%%%%%%%%%%%%%%%%%%%%%%%%%%%%%%%%%%%%%%


\tableofcontents

\newpage


%%%%%%%%%%%%%%%%%%%%%%%%%%%%%%%%%%%%%%%%%%%%%%%%%%%%%%%%%%%%%%%%%%%%%%%%%%%%%%%
%% Introducción                                                              %%
%%%%%%%%%%%%%%%%%%%%%%%%%%%%%%%%%%%%%%%%%%%%%%%%%%%%%%%%%%%%%%%%%%%%%%%%%%%%%%%


\section{Introducción}


%%%%%%%%%%%%%%%%%%%%%%%%%%%%%%%%%%%%%%%%%%%%%%%%%%%%%%%%%%%%%%%%%%%%%%%%%%%%%%%
%% Desarrollo                                                                %%
%%%%%%%%%%%%%%%%%%%%%%%%%%%%%%%%%%%%%%%%%%%%%%%%%%%%%%%%%%%%%%%%%%%%%%%%%%%%%%%


\section{Desarrollo}


%%%%%%%%%%%%%%%%%%%%%%%%%%%%%%%%%%%%%%%%%%%%%%%%%%%%%%%%%%%%%%%%%%%%%%%%%%%%%%%
%% Casos de uso                                                              %%
%%%%%%%%%%%%%%%%%%%%%%%%%%%%%%%%%%%%%%%%%%%%%%%%%%%%%%%%%%%%%%%%%%%%%%%%%%%%%%%


\section{Casos de uso}

\textbf{Registrando retiro y devolución de bicicletas no tiene la resolución del caso de uso sin conexión}

\begin{usecase}
  \ucname{Registrando retiro de bicicleta}
  \ucpre{True}
  \ucpost{Se registra el retiro de bicicleta}
  \ucactor{Personal de la estación}
\end{usecase}
\begin{usecasesteps}
  \ucstep{El sistema registra la petición de una bicicleta}{}
  \ucstep{El sistema consulta el stock}{}
  \ucstep{El sistema reserva una bicicleta del stock hasta el fin del CU}
         {\ucalt{Si no hay stock, muestra que no hay stock. Fin CU}}
  \ucstep{El personal de la estación ingresa el DNI}{}
  \ucstep{El sistema verifica que el usuario esté registrado}{}
  \ucstep{El sistema verifica que el usuario no esté penalizado}
         {\ucalt{Si el usuario no está registrado, muestra que no existe en el sistema. Fin CU}}
  \ucstep{El personal de la estación ingresa el número de la bicicleta a asignar al usuario.}
         {\ucalt{Si el usuario está penalizado, se informa que lo está. Fin CU}}
  \ucstep{El sistema registra la entrega de la bicicleta guardando DNI, ID de bicicleta y fecha y hora actual. Fin CU}{}
\end{usecasesteps}

%%%{Registrando devolución de bicicleta}
\begin{usecase}
  \ucname{Registrando devolución de bicicleta}
  \ucpre{El usuario había retirado una bicicleta}
  \ucpost{Se registra la devolución de la bicicleta entregada}
  \ucactor{Personal de la estación}
\end{usecase}
\begin{usecasesteps}
  \ucstep{El personal de la estación ingresa el número de DNI del usuario que devuelve la bicicleta.}{}
  \ucstep{El personal de la estación ingresa el ID de la bicicleta devuelta y el estado de la misma.}{}
  \ucstep{El sistema valida que el usuario que entregó la bicicleta sea el mismo que la retiró, que no haya usado la bicicleta más de una hora y que la bicicleta devuelta no esté en mal estado.}{}
  \ucstep{Si la bicicleta está en mal estado, se penaliza al usuario, o si el usuario que la devolvió no es el que la retiró, el sistema aplica penalizaciones (extiende caso de uso penalizando usuario) e informa en pantalla el motivo de la penalización.}{}
  \ucstep{El sistema registra la devolución de la bicicleta, aumenta el stock y muestra que la devolución se realizó correctamente.}{}
  \ucstep{Fin caso de uso.}{}
\end{usecasesteps}

%%%{Registrando cuenta}
\begin{usecase}
  \ucpre{True}
  \ucpost{El usuario está registrado y autenticado en el sistema}
  \ucname{Registrando cuenta}
  \ucactor{Usuario}
\end{usecase}
\begin{usecasesteps}
  \ucstep{El usuario ingresa su número de DNI, email, nombre y contraseña.}{}
  \ucstep{El sistema verifica que no esté registrado otro usuario con el email o DNI ingresado.}{}
  \ucstep{El sistema guarda los datos ingresados.}{Si los datos ingresados ya existían, mostrar que no es posible realizar el registro, y volver a 1.}
  \ucstep{El sistema muestra al usuario que el registro se realizó correctamente.}{}
  \ucstep{El sistema autentica al usuario.}{}
  \ucstep{Si lo desea, el usuario puede clickear un enlace para consultar sus multas pendientes. \\ EXTIENDE CU Consultando multas pendientes.}{}
  \ucstep{Fin caso de uso.}{}
\end{usecasesteps}

%%%{Autenticándose}
\begin{usecase}
  \ucpre{True}
  \ucpost{El usuario está autenticado en el sistema}
  \ucname{Autenticándose}
  \ucactor{Usuario}
\end{usecase}
\begin{usecasesteps}
  \ucstep{El usuario ingresa su número de DNI y su contraseña.}{}
  \ucstep{El sistema verifica que el usuario exista y que los datos ingresados sean correctos.}{}
  \ucstep{El sistema muestra al usuario que la autenticación fue satisfactoria.}{}
  \ucstep{Si lo desea, el usuario puede clickear un enlace para consultar sus multas pendientes. \\ EXTIENDE CU Consultando multas pendientes.}{}
  \ucstep{Fin caso de uso.}{}
\end{usecasesteps}

%%%{Consultando disponibilidad de bicicletas}
\begin{usecase}
  \ucpre{True}
  \ucpost{El usuario conoce la disponibilidad de la estación deseada}
  \ucname{Consultando disponibilidad de bicicletas}
  \ucactor{Usuario}
\end{usecase}
\begin{usecasesteps}
  \ucstep{El sistema muestra una lista de las estaciones a consultar por disponibilidad.}{}
  \ucstep{El usuario selecciona la estación deseada.}{}
  \ucstep{El sistema muestra la disponibilidad de la estación deseada.}{}
  \ucstep{Fin caso de uso.}{}
\end{usecasesteps}

%%%{Consultando multas pendientes}
\begin{usecase}
  \ucpre{El usuario está autenticado}
  \ucpost{El usuario conoce las multas que tiene pendientes}
  \ucname{Consultando multas pendientes}
  \ucactor{Usuario}
\end{usecase}
\begin{usecasesteps}
  \ucstep{Si se tienen multas pendientes, el sistema muestra al usuario qué tipo de multa y el importe a abonar. Caso contrario, el sistema informa que está libre de deudas.}{}
  \ucstep{Fin caso de uso.}{}
\end{usecasesteps}

%%%{Autenticándose como administrador}
\begin{usecase}
  \ucpre{True}
  \ucpost{El personal del gobierno está autenticado como administrador}
  \ucname{Consultando multas pendientes}
  \ucactor{Personal del gobierno}
\end{usecase}
\begin{usecasesteps}
  \ucstep{El personal del gobierno ingresa su usuario y su contraseña.}{}
  \ucstep{El sistema verifica que el usuario exista y que los datos ingresados sean correctos.}{}
  \ucstep{El sistema muestra al usuario que la autenticación fue satisfactoria.}{Si los datos ingresados son incorrectos, el sistema indica que la autenticación no fue satisfactoria, y vuelve a 1).}
  \ucstep{Si lo desea, el usuario puede clickear un enlace para consultar sus multas pendientes. Es extendido por CU 4. \\ 
  Si lo desea, el usuario puede clickear un enlace para registrar nuevas bicicletas. \\
  Es extendido por CU .. \\
  Si lo desea, el usuario puede clickear un enlace para registrar una nueva estación. \\
  Es extendido por CU .. \\
  Si lo desea, el usuario puede clickear un enlace para informar la eliminación de una bicicleta.}{}
  \ucstep{Fin caso de uso.}{}
\end{usecasesteps}
%%%{Registrando nuevas bicicletas}
\begin{usecase}
  \ucpre{El personal del estado está autenticado}
  \ucpost{Nuevas bicicletas registradas en el sistema y el personal del gobierno conoce los IDs asignados a las bicicletas.}
  \ucname{Registrando nuevas bicicletas}
  \ucactor{Personal del gobierno}
\end{usecase}
\begin{usecasesteps}
  \ucstep{El personal del gobierno de mar chiquita ingresa la cantidad de bicicletas nuevas.}{}
  \ucstep{El sistema registra el número de bicicletas ingresado por el personal, asignándole a cada bicicleta registrada un ID único.}{}
  \ucstep{El sistema informa al personal los IDs de las bicicletas registradas.}{}
  \ucstep{Fin caso de uso.}{}
\end{usecasesteps}

%%%{Eliminando bicicleta}
\begin{usecase}
  \ucpre{El personal del gobierno está autenticado}
  \ucpost{Una bicicleta es eliminada del sistema}
  \ucname{Informando bicicleta a eliminar}
  \ucactor{Personal del gobierno}
\end{usecase}
\begin{usecasesteps}
  \ucstep{El personal del gobierno ingresa el ID de la bicicleta a eliminar del sistema.}{}
  \ucstep{El sistema verifica que el ID de la bicicleta a eliminar corresponda a una bicicleta en el sistema.}{}
  \ucstep{El sistema muestra que la operación se realizó correctamente e informar la última ubicación de la bicicleta.}{Si el ID es incorrecto, mostrar que el ID ingresado no es válido y volver a 1).}
  \ucstep{Fin caso de uso.}{}
\end{usecasesteps}

%%%{Registrando nueva estación}
\begin{usecase}
  \ucpre{El personal del gobierno está autenticado}
  \ucpost{Se registra en el sistema la nueva estación}
  \ucname{Registrando nueva estación}
  \ucactor{Personal del gobierno}
\end{usecase}
\begin{usecasesteps}
  \ucstep{El personal del gobierno ingresa el nombre de la nueva estación y su ubicación, indicando si pertenece al centro o a la periferia.}{}
  \ucstep{El sistema verifica si ya existe una estación con el mismo nombre o la misma ubicación.}{}
  \ucstep{El sistema muestra que el ingreso de la nueva estación fue correcto.}{Si existe una estación con el mismo nombre o la misma ubicación, mostrar cuál fue el ingreso erróneo y volver a 1).}
  \ucstep{Fin caso de uso.}{}
\end{usecasesteps}

%%%{Registrando pago de multa}
\begin{usecase}
  \ucpre{La persona con el DNI provisto registraba una multa sin abonar}
  \ucpost{Se registra el cobro de la multa}
  \ucname{Registrando pago de multa}
  \ucactor{Personal de la estación}
\end{usecase}
\begin{usecasesteps}
  \ucstep{El personal de la estación ingresa el DNI de un usuario que registra multas sin abonar.}{}
  \ucstep{El sistema registra el pago de la multa y despenaliza al usuario.}{}
  \ucstep{El sistema informa que la acción fue realizada exitosamente.}{}
  \ucstep{Fin caso de uso.}{}
\end{usecasesteps}

%%%{Consultando monto a pagar para un DNI}
\begin{usecase}
  \ucpre{True}
  \ucpost{El sistema muestra las multas pendientes por pagar de un determinado DNI}
  \ucname{Consultando monto a pagar para un DNI}
  \ucactor{Personal de la estación}
\end{usecase}
\begin{usecasesteps}
  \ucstep{El personal de la estación ingresa el DNI del usuario a consultar las multas.}{}
  \ucstep{El sistema verifica que el DNI ingresado corresponda a un usuario.}{}
  \ucstep{Si existen multas por abonar, el sistema muestra que tipo de multas y el importe total. Si no existen multas, el sistema muestra que está libre de deudas.}{Si el DNI ingresado es incorrecto, mostrar mensaje de DNI equivocado y volver a 1).}
  \ucstep{Si el personal de la estación desea registrar el pago de las multas, extiende caso de uso Registrando Pago de Multa.}{}
  \ucstep{Fin caso de uso}{}
\end{usecasesteps}

%%%{Registrando bicicletas recibidas de la empresa de transporte}
\begin{usecase}
  \ucpre{Llega un camión con bicicletas y las descarga en la estación}
  \ucpost{Se registra en el sistema la llegada de las bicicletas}
  \ucname{Registrando datos de bicicletas recibidas de la empresa de transporte}
  \ucactor{Personal de la estación}
\end{usecase}
\begin{usecasesteps}
  \ucstep{El personal de la estación ingresa los ID de las bicicletas recibidas.}{}
  \ucstep{El sistema registra la ubicación de las bicicletas con los ID ingresados y actualiza el stock de la estación.}{}
  \ucstep{El sistema muestra que la operación fue realizada exitosamente.}{}
  \ucstep{Fin caso de uso.}{}
\end{usecasesteps}

%%%{Registrando bicicletas retiradas por la empresa de transporte}
\begin{usecase}
  \ucpre{El sistema dió la orden de mover bicicletas y descontó el stock de las mismas de la estación}
  \ucpost{El personal de la estación registra las bicicletas que se retirarán}
  \ucname{Registrando datos de bicicletas retiradas por la empresa de transporte}
  \ucactor{Personal de la estación}
\end{usecase}
\begin{usecasesteps}
  \ucstep{El personal de la estación ingresa los ID de las bicicletas a entregar a la empresa de transporte.}{}
  \ucstep{El sistema registra el retiro de las bicicletas con los ID ingresados.}{}
  \ucstep{El sistema muestra que la operación fue realizada exitosamente.}{}
  \ucstep{Fin caso de uso.}{}
\end{usecasesteps}

%%%{Generando mail de instrucciones para movilización de bicicletas}
\begin{usecase}
  \ucpre{True}
  \ucpost{El personal de la empresa de transporte recibe el mail con las indicaciones de cómo mover las bicicletas}
  \ucname{Generando mail de instrucciones para movilización de bicicletas}
  \ucactor{Personal de la empresa de transporte}
\end{usecase}
\begin{usecasesteps}
  \ucstep{El sistema decide qué bicicletas transportar.}{}
  \ucstep{El sistema reduce el stock de las estaciones de acuerdo a cuántas bicicletas van a ser retiradas.}{}
  \ucstep{El sistema envía mail al personal de la empresa de transporte con indicaciones de cómo mover las bicicletas.}{}
  \ucstep{El personal de la empresa de transporte recibe el mail con las indicaciones de cómo mover las bicicletas.}{}
  \ucstep{Fin caso de uso.}{}
\end{usecasesteps}

\section{Modelo conceptual}

\subsection{Modelo}

\subsection{OCL}

\end{document}